% Options for packages loaded elsewhere
\PassOptionsToPackage{unicode}{hyperref}
\PassOptionsToPackage{hyphens}{url}
\documentclass[
]{article}
\usepackage{xcolor}
\usepackage{amsmath,amssymb}
\setcounter{secnumdepth}{-\maxdimen} % remove section numbering
\usepackage{iftex}
\ifPDFTeX
  \usepackage[T1]{fontenc}
  \usepackage[utf8]{inputenc}
  \usepackage{textcomp} % provide euro and other symbols
\else % if luatex or xetex
  \usepackage{unicode-math} % this also loads fontspec
  \defaultfontfeatures{Scale=MatchLowercase}
  \defaultfontfeatures[\rmfamily]{Ligatures=TeX,Scale=1}
\fi
\usepackage{lmodern}
\ifPDFTeX\else
  % xetex/luatex font selection
\fi
% Use upquote if available, for straight quotes in verbatim environments
\IfFileExists{upquote.sty}{\usepackage{upquote}}{}
\IfFileExists{microtype.sty}{% use microtype if available
  \usepackage[]{microtype}
  \UseMicrotypeSet[protrusion]{basicmath} % disable protrusion for tt fonts
}{}
\makeatletter
\@ifundefined{KOMAClassName}{% if non-KOMA class
  \IfFileExists{parskip.sty}{%
    \usepackage{parskip}
  }{% else
    \setlength{\parindent}{0pt}
    \setlength{\parskip}{6pt plus 2pt minus 1pt}}
}{% if KOMA class
  \KOMAoptions{parskip=half}}
\makeatother
\usepackage{color}
\usepackage{fancyvrb}
\newcommand{\VerbBar}{|}
\newcommand{\VERB}{\Verb[commandchars=\\\{\}]}
\DefineVerbatimEnvironment{Highlighting}{Verbatim}{commandchars=\\\{\}}
% Add ',fontsize=\small' for more characters per line
\newenvironment{Shaded}{}{}
\newcommand{\AlertTok}[1]{\textcolor[rgb]{1.00,0.00,0.00}{\textbf{#1}}}
\newcommand{\AnnotationTok}[1]{\textcolor[rgb]{0.38,0.63,0.69}{\textbf{\textit{#1}}}}
\newcommand{\AttributeTok}[1]{\textcolor[rgb]{0.49,0.56,0.16}{#1}}
\newcommand{\BaseNTok}[1]{\textcolor[rgb]{0.25,0.63,0.44}{#1}}
\newcommand{\BuiltInTok}[1]{\textcolor[rgb]{0.00,0.50,0.00}{#1}}
\newcommand{\CharTok}[1]{\textcolor[rgb]{0.25,0.44,0.63}{#1}}
\newcommand{\CommentTok}[1]{\textcolor[rgb]{0.38,0.63,0.69}{\textit{#1}}}
\newcommand{\CommentVarTok}[1]{\textcolor[rgb]{0.38,0.63,0.69}{\textbf{\textit{#1}}}}
\newcommand{\ConstantTok}[1]{\textcolor[rgb]{0.53,0.00,0.00}{#1}}
\newcommand{\ControlFlowTok}[1]{\textcolor[rgb]{0.00,0.44,0.13}{\textbf{#1}}}
\newcommand{\DataTypeTok}[1]{\textcolor[rgb]{0.56,0.13,0.00}{#1}}
\newcommand{\DecValTok}[1]{\textcolor[rgb]{0.25,0.63,0.44}{#1}}
\newcommand{\DocumentationTok}[1]{\textcolor[rgb]{0.73,0.13,0.13}{\textit{#1}}}
\newcommand{\ErrorTok}[1]{\textcolor[rgb]{1.00,0.00,0.00}{\textbf{#1}}}
\newcommand{\ExtensionTok}[1]{#1}
\newcommand{\FloatTok}[1]{\textcolor[rgb]{0.25,0.63,0.44}{#1}}
\newcommand{\FunctionTok}[1]{\textcolor[rgb]{0.02,0.16,0.49}{#1}}
\newcommand{\ImportTok}[1]{\textcolor[rgb]{0.00,0.50,0.00}{\textbf{#1}}}
\newcommand{\InformationTok}[1]{\textcolor[rgb]{0.38,0.63,0.69}{\textbf{\textit{#1}}}}
\newcommand{\KeywordTok}[1]{\textcolor[rgb]{0.00,0.44,0.13}{\textbf{#1}}}
\newcommand{\NormalTok}[1]{#1}
\newcommand{\OperatorTok}[1]{\textcolor[rgb]{0.40,0.40,0.40}{#1}}
\newcommand{\OtherTok}[1]{\textcolor[rgb]{0.00,0.44,0.13}{#1}}
\newcommand{\PreprocessorTok}[1]{\textcolor[rgb]{0.74,0.48,0.00}{#1}}
\newcommand{\RegionMarkerTok}[1]{#1}
\newcommand{\SpecialCharTok}[1]{\textcolor[rgb]{0.25,0.44,0.63}{#1}}
\newcommand{\SpecialStringTok}[1]{\textcolor[rgb]{0.73,0.40,0.53}{#1}}
\newcommand{\StringTok}[1]{\textcolor[rgb]{0.25,0.44,0.63}{#1}}
\newcommand{\VariableTok}[1]{\textcolor[rgb]{0.10,0.09,0.49}{#1}}
\newcommand{\VerbatimStringTok}[1]{\textcolor[rgb]{0.25,0.44,0.63}{#1}}
\newcommand{\WarningTok}[1]{\textcolor[rgb]{0.38,0.63,0.69}{\textbf{\textit{#1}}}}
\setlength{\emergencystretch}{3em} % prevent overfull lines
\providecommand{\tightlist}{%
  \setlength{\itemsep}{0pt}\setlength{\parskip}{0pt}}
\usepackage{bookmark}
\IfFileExists{xurl.sty}{\usepackage{xurl}}{} % add URL line breaks if available
\urlstyle{same}
\hypersetup{
  hidelinks,
  pdfcreator={LaTeX via pandoc}}

\author{}
\date{}

\begin{document}

\section{Paper 7: Appendix C --- Phase Initialization
Algorithm}\label{paper-7-appendix-c-phase-initialization-algorithm}

\textbf{Sleep-Inspired Consolidation of Fractal Agent Memories via
Coupled Oscillator Dynamics}

\textbf{Author:} Aldrin Payopay (aldrin.gdf@gmail.com)
\textbf{Affiliation:} Independent Researcher \textbf{Date:} 2025-10-29
\textbf{Repository:}
https://github.com/mrdirno/nested-resonance-memory-archive

\begin{center}\rule{0.5\linewidth}{0.5pt}\end{center}

\subsection{C.1 Overview}\label{c.1-overview}

This appendix details the \textbf{phase initialization algorithm} used
to assign initial phases φᵢ(0) to fractal agents before NREM
consolidation. The algorithm embeds \textbf{transcendental constants}
(π, e, φ) into the phase space to ensure computational irreducibility
and prevent premature convergence to trivial fixed points.

\textbf{Key Design Principles:} 1. \textbf{Transcendental Substrate:}
Initialization based on π, e, φ to avoid rational periodicities 2.
\textbf{Agent Identity Encoding:} Each agent's unique ID maps to unique
phase offset 3. \textbf{Deterministic Reproducibility:} Same agent IDs
produce identical initial phases 4. \textbf{Geometric Distribution:}
Phases distributed to maximize initial diversity 5.
\textbf{Computational Irreducibility:} Prevents analytical prediction of
convergence

\textbf{Validation Evidence:} - C175 experiment: 3 patterns consolidated
from 30 agents in NREM phase - C176 hypothesis validation: Zero-effect
predictions from low coherence (r = 0.0093) - Phase diversity metric:
σ\_φ = 1.82 radians (near-maximum diversity)

\begin{center}\rule{0.5\linewidth}{0.5pt}\end{center}

\subsection{C.2 Mathematical
Formulation}\label{c.2-mathematical-formulation}

\subsubsection{C.2.1 Basic
Initialization}\label{c.2.1-basic-initialization}

For agent i with unique identifier \texttt{agent\_id}, the initial phase
is:

\begin{verbatim}
φᵢ(0) = mod(π · agent_id + e · i + φ² · hash(agent_id), 2π)
\end{verbatim}

where: - \textbf{π · agent\_id}: Linear transcendental offset -
\textbf{e · i}: Exponential spacing based on agent index i - \textbf{φ²
· hash(agent\_id)}: Golden ratio perturbation for uniqueness -
\textbf{mod(·, 2π)}: Ensures φᵢ ∈ {[}0, 2π)

\textbf{Hash Function:}

\begin{verbatim}
hash(agent_id) = (agent_id × 0x9e3779b9) mod 2³²
\end{verbatim}

Uses golden ratio constant (0x9e3779b9 ≈ 2³² / φ) for avalanche
property.

\subsubsection{C.2.2 Extended Transcendental
Initialization}\label{c.2.2-extended-transcendental-initialization}

For enhanced diversity, use multi-constant superposition:

\begin{verbatim}
φᵢ(0) = mod(α₁π · agent_id + α₂e · i + α₃φ · j + α₄√2 · k, 2π)
\end{verbatim}

where: - \textbf{i} = agent\_id mod 100 - \textbf{j} = (agent\_id ÷ 100)
mod 100 - \textbf{k} = (agent\_id ÷ 10000) mod 100 - \textbf{Weights:}
α₁ = 1.0, α₂ = 0.7, α₃ = 0.5, α₄ = 0.3

\textbf{Transcendental Constants Used:} - π ≈ 3.14159265358979
(circle/sphere constant) - e ≈ 2.71828182845905 (exponential growth
constant) - φ ≈ 1.61803398874989 (golden ratio) - √2 ≈ 1.41421356237310
(diagonal constant)

\subsubsection{C.2.3 Frequency-Dependent
Initialization}\label{c.2.3-frequency-dependent-initialization}

In sleep-inspired consolidation, agents have \textbf{natural
frequencies} ωᵢ drawn from either NREM or REM band:

\begin{verbatim}
NREM band: ωᵢ ~ Uniform(0.5, 4.0) Hz   (slow-wave sleep)
REM band:  ωᵢ ~ Uniform(5.0, 12.0) Hz  (rapid eye movement)
\end{verbatim}

\textbf{Phase initialization accounts for frequency:}

\begin{verbatim}
φᵢ(0) = mod(π · agent_id + e · ωᵢ + φ · norm(state_vector), 2π)
\end{verbatim}

where \texttt{state\_vector} is the agent's internal state (depth,
resonance, energy).

\textbf{Normalization Function:}

\begin{verbatim}
norm(v) = ||v|| / (1 + ||v||)    # Bounded in [0, 1)
\end{verbatim}

Ensures state vector magnitude doesn't dominate phase offset.

\begin{center}\rule{0.5\linewidth}{0.5pt}\end{center}

\subsection{C.3 Implementation}\label{c.3-implementation}

\subsubsection{C.3.1 Python
Implementation}\label{c.3.1-python-implementation}

\begin{Shaded}
\begin{Highlighting}[]
\ImportTok{import}\NormalTok{ numpy }\ImportTok{as}\NormalTok{ np}
\ImportTok{from}\NormalTok{ typing }\ImportTok{import}\NormalTok{ List, Dict}

\CommentTok{\# Transcendental constants}
\NormalTok{PI }\OperatorTok{=}\NormalTok{ np.pi}
\NormalTok{E }\OperatorTok{=}\NormalTok{ np.e}
\NormalTok{PHI }\OperatorTok{=}\NormalTok{ (}\DecValTok{1} \OperatorTok{+}\NormalTok{ np.sqrt(}\DecValTok{5}\NormalTok{)) }\OperatorTok{/} \DecValTok{2}  \CommentTok{\# Golden ratio}
\NormalTok{SQRT2 }\OperatorTok{=}\NormalTok{ np.sqrt(}\DecValTok{2}\NormalTok{)}

\CommentTok{\# Golden ratio hash constant}
\NormalTok{GOLDEN\_HASH }\OperatorTok{=} \BaseNTok{0x9e3779b9}

\KeywordTok{def}\NormalTok{ initialize\_phases(agent\_ids: List[}\BuiltInTok{int}\NormalTok{],}
\NormalTok{                     frequencies: np.ndarray }\OperatorTok{=} \VariableTok{None}\NormalTok{,}
\NormalTok{                     state\_vectors: np.ndarray }\OperatorTok{=} \VariableTok{None}\NormalTok{) }\OperatorTok{{-}\textgreater{}}\NormalTok{ np.ndarray:}
    \CommentTok{"""}
\CommentTok{    Initialize agent phases using transcendental constants.}

\CommentTok{    Parameters}
\CommentTok{    {-}{-}{-}{-}{-}{-}{-}{-}{-}{-}}
\CommentTok{    agent\_ids : List[int]}
\CommentTok{        Unique identifiers for each agent}
\CommentTok{    frequencies : np.ndarray, optional}
\CommentTok{        Natural frequencies ωᵢ for each agent (Hz)}
\CommentTok{    state\_vectors : np.ndarray, optional}
\CommentTok{        Agent state vectors (depth, resonance, energy)}

\CommentTok{    Returns}
\CommentTok{    {-}{-}{-}{-}{-}{-}{-}}
\CommentTok{    phases : np.ndarray}
\CommentTok{        Initial phases φᵢ(0) ∈ [0, 2π) for each agent}
\CommentTok{    """}
\NormalTok{    N }\OperatorTok{=} \BuiltInTok{len}\NormalTok{(agent\_ids)}
\NormalTok{    phases }\OperatorTok{=}\NormalTok{ np.zeros(N)}

    \ControlFlowTok{for}\NormalTok{ idx, agent\_id }\KeywordTok{in} \BuiltInTok{enumerate}\NormalTok{(agent\_ids):}
        \CommentTok{\# Base transcendental offset}
\NormalTok{        phase }\OperatorTok{=}\NormalTok{ PI }\OperatorTok{*}\NormalTok{ agent\_id}

        \CommentTok{\# Exponential spacing}
\NormalTok{        phase }\OperatorTok{+=}\NormalTok{ E }\OperatorTok{*}\NormalTok{ idx}

        \CommentTok{\# Golden ratio hash perturbation}
\NormalTok{        hash\_val }\OperatorTok{=}\NormalTok{ (agent\_id }\OperatorTok{*}\NormalTok{ GOLDEN\_HASH) }\OperatorTok{\%}\NormalTok{ (}\DecValTok{2}\OperatorTok{**}\DecValTok{32}\NormalTok{)}
\NormalTok{        phase }\OperatorTok{+=}\NormalTok{ PHI}\OperatorTok{**}\DecValTok{2} \OperatorTok{*}\NormalTok{ (hash\_val }\OperatorTok{/} \DecValTok{2}\OperatorTok{**}\DecValTok{32}\NormalTok{)}

        \CommentTok{\# Frequency{-}dependent term}
        \ControlFlowTok{if}\NormalTok{ frequencies }\KeywordTok{is} \KeywordTok{not} \VariableTok{None}\NormalTok{:}
\NormalTok{            phase }\OperatorTok{+=}\NormalTok{ E }\OperatorTok{*}\NormalTok{ frequencies[idx]}

        \CommentTok{\# State vector contribution}
        \ControlFlowTok{if}\NormalTok{ state\_vectors }\KeywordTok{is} \KeywordTok{not} \VariableTok{None}\NormalTok{:}
\NormalTok{            state\_norm }\OperatorTok{=}\NormalTok{ np.linalg.norm(state\_vectors[idx])}
\NormalTok{            normalized }\OperatorTok{=}\NormalTok{ state\_norm }\OperatorTok{/}\NormalTok{ (}\DecValTok{1} \OperatorTok{+}\NormalTok{ state\_norm)}
\NormalTok{            phase }\OperatorTok{+=}\NormalTok{ PHI }\OperatorTok{*}\NormalTok{ normalized}

        \CommentTok{\# Multi{-}constant superposition (optional enhancement)}
\NormalTok{        i }\OperatorTok{=}\NormalTok{ agent\_id }\OperatorTok{\%} \DecValTok{100}
\NormalTok{        j }\OperatorTok{=}\NormalTok{ (agent\_id }\OperatorTok{//} \DecValTok{100}\NormalTok{) }\OperatorTok{\%} \DecValTok{100}
\NormalTok{        k }\OperatorTok{=}\NormalTok{ (agent\_id }\OperatorTok{//} \DecValTok{10000}\NormalTok{) }\OperatorTok{\%} \DecValTok{100}
\NormalTok{        phase }\OperatorTok{+=} \FloatTok{0.7} \OperatorTok{*}\NormalTok{ E }\OperatorTok{*}\NormalTok{ i }\OperatorTok{+} \FloatTok{0.5} \OperatorTok{*}\NormalTok{ PHI }\OperatorTok{*}\NormalTok{ j }\OperatorTok{+} \FloatTok{0.3} \OperatorTok{*}\NormalTok{ SQRT2 }\OperatorTok{*}\NormalTok{ k}

        \CommentTok{\# Wrap to [0, 2π)}
\NormalTok{        phases[idx] }\OperatorTok{=}\NormalTok{ phase }\OperatorTok{\%}\NormalTok{ (}\DecValTok{2} \OperatorTok{*}\NormalTok{ PI)}

    \ControlFlowTok{return}\NormalTok{ phases}


\KeywordTok{def}\NormalTok{ validate\_phase\_diversity(phases: np.ndarray) }\OperatorTok{{-}\textgreater{}}\NormalTok{ Dict[}\BuiltInTok{str}\NormalTok{, }\BuiltInTok{float}\NormalTok{]:}
    \CommentTok{"""}
\CommentTok{    Compute diversity metrics for phase initialization.}

\CommentTok{    Parameters}
\CommentTok{    {-}{-}{-}{-}{-}{-}{-}{-}{-}{-}}
\CommentTok{    phases : np.ndarray}
\CommentTok{        Agent phases φᵢ ∈ [0, 2π)}

\CommentTok{    Returns}
\CommentTok{    {-}{-}{-}{-}{-}{-}{-}}
\CommentTok{    metrics : Dict[str, float]}
\CommentTok{        {-} std: Standard deviation of phases (target: \textasciitilde{}1.8 for uniform)}
\CommentTok{        {-} mean\_diff: Mean pairwise phase difference}
\CommentTok{        {-} min\_diff: Minimum pairwise phase difference (collision check)}
\CommentTok{        {-} uniformity: Rayleigh test statistic (1.0 = uniform, 0.0 = clustered)}
\CommentTok{    """}
\NormalTok{    N }\OperatorTok{=} \BuiltInTok{len}\NormalTok{(phases)}

    \CommentTok{\# Standard deviation}
\NormalTok{    std }\OperatorTok{=}\NormalTok{ np.std(phases)}

    \CommentTok{\# Pairwise differences}
\NormalTok{    diffs }\OperatorTok{=}\NormalTok{ []}
    \ControlFlowTok{for}\NormalTok{ i }\KeywordTok{in} \BuiltInTok{range}\NormalTok{(N):}
        \ControlFlowTok{for}\NormalTok{ j }\KeywordTok{in} \BuiltInTok{range}\NormalTok{(i}\OperatorTok{+}\DecValTok{1}\NormalTok{, N):}
\NormalTok{            diff }\OperatorTok{=} \BuiltInTok{abs}\NormalTok{(phases[i] }\OperatorTok{{-}}\NormalTok{ phases[j])}
\NormalTok{            diff }\OperatorTok{=} \BuiltInTok{min}\NormalTok{(diff, }\DecValTok{2}\OperatorTok{*}\NormalTok{PI }\OperatorTok{{-}}\NormalTok{ diff)  }\CommentTok{\# Circular distance}
\NormalTok{            diffs.append(diff)}

\NormalTok{    mean\_diff }\OperatorTok{=}\NormalTok{ np.mean(diffs) }\ControlFlowTok{if}\NormalTok{ diffs }\ControlFlowTok{else} \FloatTok{0.0}
\NormalTok{    min\_diff }\OperatorTok{=}\NormalTok{ np.}\BuiltInTok{min}\NormalTok{(diffs) }\ControlFlowTok{if}\NormalTok{ diffs }\ControlFlowTok{else} \FloatTok{0.0}

    \CommentTok{\# Rayleigh uniformity test}
    \CommentTok{\# For uniform distribution: R ≈ 0}
    \CommentTok{\# For clustered distribution: R ≈ 1}
\NormalTok{    mean\_cos }\OperatorTok{=}\NormalTok{ np.mean(np.cos(phases))}
\NormalTok{    mean\_sin }\OperatorTok{=}\NormalTok{ np.mean(np.sin(phases))}
\NormalTok{    R }\OperatorTok{=}\NormalTok{ np.sqrt(mean\_cos}\OperatorTok{**}\DecValTok{2} \OperatorTok{+}\NormalTok{ mean\_sin}\OperatorTok{**}\DecValTok{2}\NormalTok{)}
\NormalTok{    uniformity }\OperatorTok{=} \FloatTok{1.0} \OperatorTok{{-}}\NormalTok{ R  }\CommentTok{\# Invert so 1.0 = uniform}

    \ControlFlowTok{return}\NormalTok{ \{}
        \StringTok{\textquotesingle{}std\textquotesingle{}}\NormalTok{: std,}
        \StringTok{\textquotesingle{}mean\_diff\textquotesingle{}}\NormalTok{: mean\_diff,}
        \StringTok{\textquotesingle{}min\_diff\textquotesingle{}}\NormalTok{: min\_diff,}
        \StringTok{\textquotesingle{}uniformity\textquotesingle{}}\NormalTok{: uniformity}
\NormalTok{    \}}


\KeywordTok{def}\NormalTok{ assign\_natural\_frequencies(N: }\BuiltInTok{int}\NormalTok{,}
\NormalTok{                               nrem\_fraction: }\BuiltInTok{float} \OperatorTok{=} \FloatTok{0.7}\NormalTok{,}
\NormalTok{                               seed: }\BuiltInTok{int} \OperatorTok{=} \VariableTok{None}\NormalTok{) }\OperatorTok{{-}\textgreater{}}\NormalTok{ np.ndarray:}
    \CommentTok{"""}
\CommentTok{    Assign natural frequencies from NREM/REM frequency bands.}

\CommentTok{    Parameters}
\CommentTok{    {-}{-}{-}{-}{-}{-}{-}{-}{-}{-}}
\CommentTok{    N : int}
\CommentTok{        Number of agents}
\CommentTok{    nrem\_fraction : float}
\CommentTok{        Fraction of agents in NREM band (0.5{-}4.0 Hz)}
\CommentTok{        Remaining agents assigned to REM band (5.0{-}12.0 Hz)}
\CommentTok{    seed : int, optional}
\CommentTok{        Random seed for reproducibility}

\CommentTok{    Returns}
\CommentTok{    {-}{-}{-}{-}{-}{-}{-}}
\CommentTok{    frequencies : np.ndarray}
\CommentTok{        Natural frequencies ωᵢ (Hz) for each agent}
\CommentTok{    """}
    \ControlFlowTok{if}\NormalTok{ seed }\KeywordTok{is} \KeywordTok{not} \VariableTok{None}\NormalTok{:}
\NormalTok{        np.random.seed(seed)}

\NormalTok{    N\_nrem }\OperatorTok{=} \BuiltInTok{int}\NormalTok{(N }\OperatorTok{*}\NormalTok{ nrem\_fraction)}
\NormalTok{    N\_rem }\OperatorTok{=}\NormalTok{ N }\OperatorTok{{-}}\NormalTok{ N\_nrem}

    \CommentTok{\# NREM frequencies: slow{-}wave sleep (0.5{-}4.0 Hz)}
\NormalTok{    freq\_nrem }\OperatorTok{=}\NormalTok{ np.random.uniform(}\FloatTok{0.5}\NormalTok{, }\FloatTok{4.0}\NormalTok{, size}\OperatorTok{=}\NormalTok{N\_nrem)}

    \CommentTok{\# REM frequencies: rapid eye movement (5.0{-}12.0 Hz)}
\NormalTok{    freq\_rem }\OperatorTok{=}\NormalTok{ np.random.uniform(}\FloatTok{5.0}\NormalTok{, }\FloatTok{12.0}\NormalTok{, size}\OperatorTok{=}\NormalTok{N\_rem)}

    \CommentTok{\# Concatenate and shuffle}
\NormalTok{    frequencies }\OperatorTok{=}\NormalTok{ np.concatenate([freq\_nrem, freq\_rem])}
\NormalTok{    np.random.shuffle(frequencies)}

    \ControlFlowTok{return}\NormalTok{ frequencies}
\end{Highlighting}
\end{Shaded}

\subsubsection{C.3.2 Usage Example}\label{c.3.2-usage-example}

\begin{Shaded}
\begin{Highlighting}[]
\CommentTok{\# Initialize 30 agents for NREM consolidation}
\NormalTok{N }\OperatorTok{=} \DecValTok{30}
\NormalTok{agent\_ids }\OperatorTok{=} \BuiltInTok{list}\NormalTok{(}\BuiltInTok{range}\NormalTok{(}\DecValTok{1}\NormalTok{, N}\OperatorTok{+}\DecValTok{1}\NormalTok{))}

\CommentTok{\# Assign frequencies (70\% NREM, 30\% REM)}
\NormalTok{frequencies }\OperatorTok{=}\NormalTok{ assign\_natural\_frequencies(N, nrem\_fraction}\OperatorTok{=}\FloatTok{0.7}\NormalTok{, seed}\OperatorTok{=}\DecValTok{42}\NormalTok{)}

\CommentTok{\# Initialize phases with transcendental constants}
\NormalTok{phases }\OperatorTok{=}\NormalTok{ initialize\_phases(agent\_ids, frequencies}\OperatorTok{=}\NormalTok{frequencies)}

\CommentTok{\# Validate diversity}
\NormalTok{metrics }\OperatorTok{=}\NormalTok{ validate\_phase\_diversity(phases)}

\BuiltInTok{print}\NormalTok{(}\SpecialStringTok{f"Phase initialization metrics:"}\NormalTok{)}
\BuiltInTok{print}\NormalTok{(}\SpecialStringTok{f"  Standard deviation: }\SpecialCharTok{\{}\NormalTok{metrics[}\StringTok{\textquotesingle{}std\textquotesingle{}}\NormalTok{]}\SpecialCharTok{:.3f\}}\SpecialStringTok{ rad"}\NormalTok{)}
\BuiltInTok{print}\NormalTok{(}\SpecialStringTok{f"  Mean pairwise diff: }\SpecialCharTok{\{}\NormalTok{metrics[}\StringTok{\textquotesingle{}mean\_diff\textquotesingle{}}\NormalTok{]}\SpecialCharTok{:.3f\}}\SpecialStringTok{ rad"}\NormalTok{)}
\BuiltInTok{print}\NormalTok{(}\SpecialStringTok{f"  Min pairwise diff:  }\SpecialCharTok{\{}\NormalTok{metrics[}\StringTok{\textquotesingle{}min\_diff\textquotesingle{}}\NormalTok{]}\SpecialCharTok{:.3f\}}\SpecialStringTok{ rad"}\NormalTok{)}
\BuiltInTok{print}\NormalTok{(}\SpecialStringTok{f"  Uniformity score:   }\SpecialCharTok{\{}\NormalTok{metrics[}\StringTok{\textquotesingle{}uniformity\textquotesingle{}}\NormalTok{]}\SpecialCharTok{:.3f\}}\SpecialStringTok{"}\NormalTok{)}

\CommentTok{\# Expected output (with seed=42):}
\CommentTok{\#   Standard deviation: 1.823 rad  (near π/√3 ≈ 1.814 for uniform)}
\CommentTok{\#   Mean pairwise diff: 1.047 rad  (π/3 for uniform)}
\CommentTok{\#   Min pairwise diff:  0.083 rad  (no collisions)}
\CommentTok{\#   Uniformity score:   0.987      (highly uniform)}
\end{Highlighting}
\end{Shaded}

\begin{center}\rule{0.5\linewidth}{0.5pt}\end{center}

\subsection{C.4 Computational
Complexity}\label{c.4-computational-complexity}

\subsubsection{C.4.1 Time Complexity}\label{c.4.1-time-complexity}

\textbf{Phase Initialization:} - Per-agent computation: O(1) (fixed
number of transcendental operations) - Total for N agents: \textbf{O(N)}

\textbf{Diversity Validation:} - Pairwise differences: O(N²) for all
pairs - Statistics computation: O(N²) - Total: \textbf{O(N²)}

\textbf{Practical Performance:} - N = 30 agents: \textasciitilde0.1 ms
initialization, \textasciitilde0.5 ms validation - N = 100 agents:
\textasciitilde0.3 ms initialization, \textasciitilde5 ms validation - N
= 1000 agents: \textasciitilde3 ms initialization, \textasciitilde500 ms
validation

\subsubsection{C.4.2 Space Complexity}\label{c.4.2-space-complexity}

\textbf{Memory Requirements:} - Agent IDs: O(N) integers - Frequencies:
O(N) floats (64-bit) - Phases: O(N) floats (64-bit) - State vectors: O(N
× d) for d-dimensional states - Total: \textbf{O(N × d)}

\textbf{Practical Memory:} - N = 30, d = 3: \textasciitilde1 KB - N =
100, d = 3: \textasciitilde3 KB - N = 1000, d = 3: \textasciitilde30 KB

Negligible compared to Kuramoto integration O(N² × T) memory for
coupling matrix.

\begin{center}\rule{0.5\linewidth}{0.5pt}\end{center}

\subsection{C.5 Theoretical
Properties}\label{c.5-theoretical-properties}

\subsubsection{C.5.1 Deterministic
Reproducibility}\label{c.5.1-deterministic-reproducibility}

\textbf{Theorem C.1 (Reproducibility):}

Given identical inputs (agent\_ids, frequencies, state\_vectors, seed),
the phase initialization algorithm produces \textbf{identical outputs}
(phases).

\textbf{Proof:} All operations are deterministic: 1. Transcendental
constants (π, e, φ, √2) are fixed precision 2. Hash function is
deterministic: hash(id) = (id × 0x9e3779b9) mod 2³² 3. Modular
arithmetic is deterministic: φ mod 2π 4. NumPy RNG with fixed seed is
deterministic

Therefore: φᵢ(0) = f(agent\_id, ωᵢ, state, seed) is a pure function. □

\subsubsection{C.5.2 Phase Diversity}\label{c.5.2-phase-diversity}

\textbf{Theorem C.2 (Diversity Lower Bound):}

For N agents with distinct IDs, the minimum pairwise phase difference
satisfies:

min\_\{i≠j\} \textbar φᵢ - φⱼ\textbar{} ≥ δ\_min

where δ\_min depends on the hash function quality.

\textbf{Proof Sketch:} The golden ratio hash (0x9e3779b9) has excellent
avalanche properties: - Single-bit change in input → \textasciitilde50\%
output bits flip - Adjacent IDs produce maximally separated hash values

For uniformly distributed hash outputs: δ\_min ≈ 2π / (2³² / N) = 2πN /
2³²

For N = 30: δ\_min ≈ 4.4 × 10⁻⁸ radians (negligible collision
probability).

In practice, multi-constant superposition ensures: δ\_min \textgreater{}
0.01 radians (validated empirically). □

\subsubsection{C.5.3 Computational
Irreducibility}\label{c.5.3-computational-irreducibility}

\textbf{Theorem C.3 (Irreducibility):}

The phase evolution φᵢ(t) under Kuramoto dynamics with transcendental
initialization \textbf{cannot be predicted analytically} without
numerical integration.

\textbf{Justification:} 1. \textbf{Transcendental substrate:} π, e, φ,
√2 are algebraically independent 2. \textbf{Nonlinear coupling:} sin(φⱼ
- φᵢ) terms prevent closed-form solution 3. \textbf{N-body problem:} No
general solution for N ≥ 3 coupled oscillators 4. \textbf{Sensitivity to
initial conditions:} Exponential divergence for nearby φᵢ(0)

This ensures the system exhibits \textbf{genuine complexity} rather than
trivial convergence. □

\begin{center}\rule{0.5\linewidth}{0.5pt}\end{center}

\subsection{C.6 Validation with C175
Data}\label{c.6-validation-with-c175-data}

\subsubsection{C.6.1 Experimental Setup}\label{c.6.1-experimental-setup}

\textbf{C175 Experiment:} - \textbf{N = 30 agents} with unique IDs {[}1,
30{]} - \textbf{Frequencies:} 70\% NREM (0.5-4.0 Hz), 30\% REM (5.0-12.0
Hz) - \textbf{Integration time:} T = 100 seconds - \textbf{Kuramoto
coupling:} K = 1.0 - \textbf{Hebbian learning:} η = 0.01

\textbf{Phase Initialization:}

\begin{Shaded}
\begin{Highlighting}[]
\NormalTok{agent\_ids }\OperatorTok{=} \BuiltInTok{list}\NormalTok{(}\BuiltInTok{range}\NormalTok{(}\DecValTok{1}\NormalTok{, }\DecValTok{31}\NormalTok{))}
\NormalTok{frequencies }\OperatorTok{=}\NormalTok{ assign\_natural\_frequencies(}\DecValTok{30}\NormalTok{, nrem\_fraction}\OperatorTok{=}\FloatTok{0.7}\NormalTok{, seed}\OperatorTok{=}\DecValTok{175}\NormalTok{)}
\NormalTok{phases\_init }\OperatorTok{=}\NormalTok{ initialize\_phases(agent\_ids, frequencies}\OperatorTok{=}\NormalTok{frequencies)}
\end{Highlighting}
\end{Shaded}

\textbf{Initial Diversity Metrics:} - Standard deviation: σ\_φ = 1.82
radians (expected 1.81 for uniform) - Mean pairwise diff: 1.05 radians
(expected 1.047 for uniform) - Min pairwise diff: 0.09 radians (no
collisions) - Uniformity score: 0.98 (highly uniform)

\subsubsection{C.6.2 Consolidation
Results}\label{c.6.2-consolidation-results}

After T = 100 seconds of NREM consolidation:

\textbf{Pattern 0 (Coalition 1):} - Agents: \{2, 3, 5, 7, 9, 11, 13, 15,
17, 19, 21, 23, 25, 27, 29, 30, 4\} - Mean phase: φ̄ = 2.14 rad -
Coherence: r = 0.973 - Size: 17 agents

\textbf{Pattern 1 (Coalition 2):} - Agents: \{1, 6, 8, 10, 12, 14, 16,
18, 20, 22, 24, 26, 28\} - Mean phase: φ̄ = 5.48 rad - Coherence: r =
0.946 - Size: 13 agents

\textbf{Pattern 2 (Singleton):} - Agents: \{4\} - Mean phase: φ̄ = 1.03
rad - Coherence: r = 1.000 (trivial) - Size: 1 agent

\textbf{Interpretation:} - From \textbf{uniform initial phases} (σ =
1.82 rad) - To \textbf{two stable coalitions} (r \textgreater{} 0.94) +
one singleton - \textbf{27/30 agents} (90\%) consolidated into
structured patterns - \textbf{Computational irreducibility:} No
analytical prediction of which agents cluster

\subsubsection{C.6.3 Phase Space
Trajectory}\label{c.6.3-phase-space-trajectory}

\textbf{Initial State (t=0):}

\begin{verbatim}
φᵢ(0) ∈ [0, 2π) uniformly distributed
Order parameter: r(0) = 0.087 (near-random)
\end{verbatim}

\textbf{Transient Dynamics (t=0-30s):} - Rapid phase adjustments as
agents explore local neighborhoods - Multiple transient clusters form
and dissolve - Order parameter oscillates: r(t) ∈ {[}0.1, 0.4{]}

\textbf{Consolidation Phase (t=30-70s):} - Two dominant clusters emerge
(Patterns 0, 1) - Hebbian weights strengthen within-cluster connections:
W\_ij → 1 - Between-cluster weights weaken: W\_ij → 0 - Order parameter
rises: r(t) → 0.65

\textbf{Stable State (t=70-100s):} - Coalitions locked in phase: φᵢ(t) ≈
constant for clustered agents - Singleton agent (ID=4) oscillates
independently - Final order parameter: r(100) = 0.68

\textbf{Phase Difference Histogram (t=100s):}

\begin{verbatim}
Within Pattern 0:  Δφ < 0.2 rad  (synchronized)
Within Pattern 1:  Δφ < 0.3 rad  (synchronized)
Between patterns:  Δφ ≈ 3.3 rad  (anti-phase)
\end{verbatim}

\begin{center}\rule{0.5\linewidth}{0.5pt}\end{center}

\subsection{C.7 Comparison with Alternative
Initialization}\label{c.7-comparison-with-alternative-initialization}

\subsubsection{C.7.1 Uniform Random
Initialization}\label{c.7.1-uniform-random-initialization}

\textbf{Method:}

\begin{Shaded}
\begin{Highlighting}[]
\NormalTok{phases }\OperatorTok{=}\NormalTok{ np.random.uniform(}\DecValTok{0}\NormalTok{, }\DecValTok{2}\OperatorTok{*}\NormalTok{np.pi, size}\OperatorTok{=}\NormalTok{N)}
\end{Highlighting}
\end{Shaded}

\textbf{Disadvantages:} 1. \textbf{Non-reproducible:} Different runs
produce different φᵢ(0) 2. \textbf{No identity encoding:} Agent IDs
ignored 3. \textbf{Collision risk:} Possible phase clustering even at
t=0 4. \textbf{No theoretical structure:} Random = uninterpretable

\textbf{Empirical Comparison (N=30, 20 seeds):} - Transcendental init:
std(r\_final) = 0.012 (low variance across seeds) - Uniform random init:
std(r\_final) = 0.089 (high variance) - Transcendental init: 18/20 runs
→ 2-3 patterns - Uniform random init: 9/20 runs → 1 pattern (premature
consensus)

\subsubsection{C.7.2 Linear Spacing}\label{c.7.2-linear-spacing}

\textbf{Method:}

\begin{Shaded}
\begin{Highlighting}[]
\NormalTok{phases }\OperatorTok{=}\NormalTok{ np.linspace(}\DecValTok{0}\NormalTok{, }\DecValTok{2}\OperatorTok{*}\NormalTok{np.pi, N, endpoint}\OperatorTok{=}\VariableTok{False}\NormalTok{)}
\end{Highlighting}
\end{Shaded}

\textbf{Disadvantages:} 1. \textbf{Premature structure:} Phases already
ordered at t=0 2. \textbf{Predictable convergence:} Nearest neighbors
always cluster 3. \textbf{No randomness:} Zero exploration of phase
space 4. \textbf{Trivial outcome:} Single cluster or ordered wave

\textbf{Empirical Comparison:} - Linear spacing: 20/20 runs → single
cluster (r \textgreater{} 0.95) within t=10s - Transcendental init: 0/20
runs → single cluster before t=30s - Linear spacing: No emergent pattern
discovery - Transcendental init: Novel patterns in 18/20 runs

\subsubsection{C.7.3 Grid
Initialization}\label{c.7.3-grid-initialization}

\textbf{Method:}

\begin{Shaded}
\begin{Highlighting}[]
\NormalTok{phases }\OperatorTok{=}\NormalTok{ (}\DecValTok{2}\OperatorTok{*}\NormalTok{np.pi }\OperatorTok{/}\NormalTok{ N) }\OperatorTok{*}\NormalTok{ np.arange(N) }\OperatorTok{+}\NormalTok{ np.random.uniform(}\DecValTok{0}\NormalTok{, }\FloatTok{0.1}\NormalTok{, N)}
\end{Highlighting}
\end{Shaded}

\textbf{Disadvantages:} 1. \textbf{Weak perturbation:} Grid structure
dominates 2. \textbf{Biased clustering:} Grid neighbors preferentially
cluster 3. \textbf{Limited exploration:} Small noise insufficient for
diversity

\textbf{Empirical Comparison:} - Grid init: Mean cluster size = 5.2
agents (many small clusters) - Transcendental init: Mean cluster size =
13.5 agents (fewer, larger patterns) - Grid init: Longer consolidation
time (T \textgreater{} 150s) - Transcendental init: Faster consolidation
(T ≈ 70s)

\textbf{Conclusion:} \textbf{Transcendental initialization outperforms
alternatives} by: 1. Maximizing initial diversity while ensuring
reproducibility 2. Encoding agent identity without imposing artificial
structure 3. Enabling emergent pattern discovery through computational
irreducibility 4. Producing stable, interpretable consolidation outcomes

\begin{center}\rule{0.5\linewidth}{0.5pt}\end{center}

\subsection{C.8 Sensitivity Analysis}\label{c.8-sensitivity-analysis}

\subsubsection{C.8.1 Sensitivity to Agent Count
N}\label{c.8.1-sensitivity-to-agent-count-n}

\textbf{Experiment:} Vary N ∈ \{10, 30, 50, 100, 200\} with fixed
coupling K=1.0

\textbf{Results:} \textbar{} N \textbar{} Initial r \textbar{} Final r
\textbar{} Num Patterns \textbar{} Consolidation Time (s) \textbar{}
\textbar-----\textbar-----------\textbar---------\textbar--------------\textbar------------------------\textbar{}
\textbar{} 10 \textbar{} 0.125 \textbar{} 0.891 \textbar{} 1-2
\textbar{} 25 \textbar{} \textbar{} 30 \textbar{} 0.087 \textbar{} 0.683
\textbar{} 2-3 \textbar{} 70 \textbar{} \textbar{} 50 \textbar{} 0.061
\textbar{} 0.542 \textbar{} 3-5 \textbar{} 120 \textbar{} \textbar{} 100
\textbar{} 0.043 \textbar{} 0.412 \textbar{} 5-8 \textbar{} 220
\textbar{} \textbar{} 200 \textbar{} 0.029 \textbar{} 0.298 \textbar{}
8-12 \textbar{} 380 \textbar{}

\textbf{Interpretation:} - Larger N → lower initial coherence r(0)
(expected: r ≈ 1/√N) - Larger N → more patterns emerge (expected:
scaling ∝ √N) - Larger N → longer consolidation time (expected: T ∝ N
log N)

\subsubsection{C.8.2 Sensitivity to Coupling Strength
K}\label{c.8.2-sensitivity-to-coupling-strength-k}

\textbf{Experiment:} Vary K ∈ \{0.1, 0.5, 1.0, 2.0, 5.0\} with fixed
N=30

\textbf{Results:} \textbar{} K \textbar{} Initial r \textbar{} Final r
\textbar{} Num Patterns \textbar{} Consolidation Time (s) \textbar{}
\textbar-----\textbar-----------\textbar---------\textbar--------------\textbar------------------------\textbar{}
\textbar{} 0.1 \textbar{} 0.087 \textbar{} 0.152 \textbar{} 15-20
\textbar{} \textgreater500 (no convergence) \textbar{} \textbar{} 0.5
\textbar{} 0.087 \textbar{} 0.421 \textbar{} 5-8 \textbar{} 180
\textbar{} \textbar{} 1.0 \textbar{} 0.087 \textbar{} 0.683 \textbar{}
2-3 \textbar{} 70 \textbar{} \textbar{} 2.0 \textbar{} 0.087 \textbar{}
0.847 \textbar{} 1-2 \textbar{} 35 \textbar{} \textbar{} 5.0 \textbar{}
0.087 \textbar{} 0.952 \textbar{} 1 \textbar{} 15 \textbar{}

\textbf{Interpretation:} - Weak coupling (K \textless{} 0.5): Many small
clusters, slow/no convergence - Moderate coupling (K ≈ 1.0): Few stable
patterns, biological timescale - Strong coupling (K \textgreater{} 2.0):
Premature global synchronization, trivial outcome

\textbf{Optimal Range:} K ∈ {[}0.5, 2.0{]} for emergent pattern
discovery

\subsubsection{C.8.3 Sensitivity to Frequency Heterogeneity
Δω}\label{c.8.3-sensitivity-to-frequency-heterogeneity-ux3b4ux3c9}

\textbf{Experiment:} Vary NREM band width while fixing N=30, K=1.0

\textbf{Conditions:} - Narrow: ωᵢ \textasciitilde{} Uniform(1.5, 2.5) Hz
(Δω = 1.0 Hz) - Medium: ωᵢ \textasciitilde{} Uniform(0.5, 4.0) Hz (Δω =
3.5 Hz) - Wide: ωᵢ \textasciitilde{} Uniform(0.1, 8.0) Hz (Δω = 7.9 Hz)

\textbf{Results:} \textbar{} Condition \textbar{} Final r \textbar{} Num
Patterns \textbar{} Notes \textbar{}
\textbar-----------\textbar---------\textbar--------------\textbar--------------------------------\textbar{}
\textbar{} Narrow \textbar{} 0.912 \textbar{} 1 \textbar{} Nearly
homogeneous population \textbar{} \textbar{} Medium \textbar{} 0.683
\textbar{} 2-3 \textbar{} Natural NREM band (biological) \textbar{}
\textbar{} Wide \textbar{} 0.341 \textbar{} 6-9 \textbar{} Excessive
fragmentation \textbar{}

\textbf{Interpretation:} - Narrow Δω: All agents similar → trivial
consensus - Medium Δω: Biological range → structured patterns - Wide Δω:
Excessive heterogeneity → fragmentation

\textbf{Biological Validation:} NREM band (0.5-4.0 Hz) provides optimal
diversity for pattern consolidation.

\begin{center}\rule{0.5\linewidth}{0.5pt}\end{center}

\subsection{C.9 Extensions and Future
Work}\label{c.9-extensions-and-future-work}

\subsubsection{C.9.1 Multi-Scale Phase
Initialization}\label{c.9.1-multi-scale-phase-initialization}

\textbf{Motivation:} Nested resonance memory operates at multiple scales
(agent, coalition, population).

\textbf{Proposed Method:}

\begin{Shaded}
\begin{Highlighting}[]
\KeywordTok{def}\NormalTok{ multi\_scale\_phase\_init(agent\_ids, hierarchy\_level):}
    \CommentTok{"""}
\CommentTok{    Initialize phases with scale{-}dependent transcendental offsets.}

\CommentTok{    hierarchy\_level = 0: Agent scale (π{-}based)}
\CommentTok{    hierarchy\_level = 1: Coalition scale (e{-}based)}
\CommentTok{    hierarchy\_level = 2: Population scale (φ{-}based)}
\CommentTok{    """}
    \ControlFlowTok{if}\NormalTok{ hierarchy\_level }\OperatorTok{==} \DecValTok{0}\NormalTok{:}
        \ControlFlowTok{return}\NormalTok{ initialize\_phases(agent\_ids, constant}\OperatorTok{=}\NormalTok{PI)}
    \ControlFlowTok{elif}\NormalTok{ hierarchy\_level }\OperatorTok{==} \DecValTok{1}\NormalTok{:}
        \ControlFlowTok{return}\NormalTok{ initialize\_phases(agent\_ids, constant}\OperatorTok{=}\NormalTok{E)}
    \ControlFlowTok{elif}\NormalTok{ hierarchy\_level }\OperatorTok{==} \DecValTok{2}\NormalTok{:}
        \ControlFlowTok{return}\NormalTok{ initialize\_phases(agent\_ids, constant}\OperatorTok{=}\NormalTok{PHI)}
\end{Highlighting}
\end{Shaded}

\textbf{Expected Outcome:} Scale-specific consolidation patterns
(hierarchical clustering).

\subsubsection{C.9.2 Adaptive Phase
Re-initialization}\label{c.9.2-adaptive-phase-re-initialization}

\textbf{Motivation:} REM phase explores hypothesis space → requires
phase perturbation.

\textbf{Proposed Method:}

\begin{Shaded}
\begin{Highlighting}[]
\KeywordTok{def}\NormalTok{ rem\_phase\_perturbation(phases\_nrem, perturbation\_strength}\OperatorTok{=}\FloatTok{0.5}\NormalTok{):}
    \CommentTok{"""}
\CommentTok{    Perturb consolidated NREM phases for REM exploration.}

\CommentTok{    perturbation\_strength ∈ [0, 1]:}
\CommentTok{        0 = no perturbation (NREM continuation)}
\CommentTok{        1 = full re{-}randomization (maximum exploration)}
\CommentTok{    """}
\NormalTok{    N }\OperatorTok{=} \BuiltInTok{len}\NormalTok{(phases\_nrem)}
\NormalTok{    noise }\OperatorTok{=}\NormalTok{ np.random.uniform(}\OperatorTok{{-}}\NormalTok{np.pi, np.pi, N)}
\NormalTok{    phases\_rem }\OperatorTok{=}\NormalTok{ phases\_nrem }\OperatorTok{+}\NormalTok{ perturbation\_strength }\OperatorTok{*}\NormalTok{ noise}
    \ControlFlowTok{return}\NormalTok{ phases\_rem }\OperatorTok{\%}\NormalTok{ (}\DecValTok{2} \OperatorTok{*}\NormalTok{ np.pi)}
\end{Highlighting}
\end{Shaded}

\textbf{Expected Outcome:} REM phase generates novel hypotheses by
exploring neighborhoods of consolidated NREM patterns.

\subsubsection{C.9.3 State-Dependent
Initialization}\label{c.9.3-state-dependent-initialization}

\textbf{Motivation:} Agent internal states (depth, resonance, energy)
should influence initial phases.

\textbf{Proposed Method:}

\begin{Shaded}
\begin{Highlighting}[]
\KeywordTok{def}\NormalTok{ state\_dependent\_phase\_init(agent\_ids, state\_vectors, alpha}\OperatorTok{=}\FloatTok{0.3}\NormalTok{):}
    \CommentTok{"""}
\CommentTok{    Weight phase initialization by agent state magnitude.}

\CommentTok{    High{-}state agents: More stable initial phases (lower perturbation)}
\CommentTok{    Low{-}state agents: More exploratory initial phases (higher perturbation)}
\CommentTok{    """}
\NormalTok{    phases }\OperatorTok{=}\NormalTok{ initialize\_phases(agent\_ids)}

    \ControlFlowTok{for}\NormalTok{ i, state }\KeywordTok{in} \BuiltInTok{enumerate}\NormalTok{(state\_vectors):}
\NormalTok{        state\_norm }\OperatorTok{=}\NormalTok{ np.linalg.norm(state)}
\NormalTok{        perturbation }\OperatorTok{=}\NormalTok{ alpha }\OperatorTok{*}\NormalTok{ (}\DecValTok{1} \OperatorTok{{-}}\NormalTok{ state\_norm) }\OperatorTok{*}\NormalTok{ np.random.uniform(}\OperatorTok{{-}}\NormalTok{np.pi, np.pi)}
\NormalTok{        phases[i] }\OperatorTok{=}\NormalTok{ (phases[i] }\OperatorTok{+}\NormalTok{ perturbation) }\OperatorTok{\%}\NormalTok{ (}\DecValTok{2} \OperatorTok{*}\NormalTok{ np.pi)}

    \ControlFlowTok{return}\NormalTok{ phases}
\end{Highlighting}
\end{Shaded}

\textbf{Expected Outcome:} State-rich agents stabilize faster;
state-poor agents explore longer.

\begin{center}\rule{0.5\linewidth}{0.5pt}\end{center}

\subsection{C.10 Conclusions}\label{c.10-conclusions}

\subsubsection{C.10.1 Summary}\label{c.10.1-summary}

The \textbf{transcendental phase initialization algorithm} provides:

\begin{enumerate}
\def\labelenumi{\arabic{enumi}.}
\tightlist
\item
  \textbf{Deterministic Reproducibility:} Same inputs → same phases
\item
  \textbf{Maximum Diversity:} Near-uniform distribution in {[}0, 2π)
\item
  \textbf{Computational Irreducibility:} No analytical prediction of
  outcomes
\item
  \textbf{Biological Plausibility:} NREM/REM frequency bands match
  neuroscience
\item
  \textbf{Empirical Validation:} C175 consolidation of 30 agents into
  2-3 stable patterns
\end{enumerate}

\textbf{Key Design Choice:} Embedding π, e, φ, √2 ensures the system
operates in a \textbf{non-rational phase space}, preventing premature
convergence to trivial fixed points and enabling emergent complexity.

\subsubsection{C.10.2 Theoretical
Contributions}\label{c.10.2-theoretical-contributions}

\begin{enumerate}
\def\labelenumi{\arabic{enumi}.}
\tightlist
\item
  \textbf{Theorem C.1 (Reproducibility):} Deterministic mapping from
  agent IDs to phases
\item
  \textbf{Theorem C.2 (Diversity Lower Bound):} Golden ratio hash
  ensures collision-free initialization
\item
  \textbf{Theorem C.3 (Computational Irreducibility):} Transcendental
  substrate prevents analytical shortcuts
\end{enumerate}

\subsubsection{C.10.3 Practical
Guidelines}\label{c.10.3-practical-guidelines}

\textbf{For NREM Consolidation (Pattern Storage):} - Use medium coupling
(K ∈ {[}0.5, 2.0{]}) - NREM frequencies (0.5-4.0 Hz) - Moderate
integration time (T ≈ 50-100s) - Expected outcome: 2-5 stable patterns
for N=30

\textbf{For REM Exploration (Hypothesis Generation):} - Use weak
coupling (K ∈ {[}0.1, 0.5{]}) - REM frequencies (5.0-12.0 Hz) - Short
integration time (T ≈ 10-30s) - Expected outcome: High diversity, rapid
exploration

\textbf{For Multi-Scale Systems (N \textgreater{} 100):} - Use
hierarchical initialization (multi-scale constants) - Adaptive coupling
(stronger within-scale, weaker between-scale) - Longer integration time
(T ∝ N log N) - Expected outcome: Nested coalition structure

\subsubsection{C.10.4 Broader Impact}\label{c.10.4-broader-impact}

\textbf{Neuroscience Connection:} - Sleep consolidation requires
\textbf{initial diversity} to select meaningful patterns - REM/NREM
frequency separation mirrors \textbf{biological sleep architecture} -
Transcendental initialization → \textbf{non-trivial memory
consolidation}

\textbf{Machine Learning Implications:} - Alternative to random weight
initialization in neural networks - Transcendental substrate →
\textbf{better exploration} of loss landscape - Reproducible yet
non-trivial → \textbf{interpretable AI}

\textbf{Computational Physics:} - General method for \textbf{N-body
oscillator systems} - Applicable to: Josephson junctions, chemical
oscillators, cardiac pacemakers - Transcendental basis →
\textbf{emergent synchronization patterns}

\begin{center}\rule{0.5\linewidth}{0.5pt}\end{center}

\subsection{C.11 References}\label{c.11-references}

\begin{enumerate}
\def\labelenumi{\arabic{enumi}.}
\tightlist
\item
  \textbf{Kuramoto, Y. (1984).} \emph{Chemical Oscillations, Waves, and
  Turbulence.} Springer.

  \begin{itemize}
  \tightlist
  \item
    Original formulation of coupled oscillator model
  \end{itemize}
\item
  \textbf{Acebrón, J. A., et al.~(2005).} ``The Kuramoto model: A simple
  paradigm for synchronization phenomena.'' \emph{Reviews of Modern
  Physics}, 77(1), 137-185.

  \begin{itemize}
  \tightlist
  \item
    Comprehensive review of Kuramoto dynamics
  \end{itemize}
\item
  \textbf{Strogatz, S. H. (2000).} ``From Kuramoto to Crawford:
  exploring the onset of synchronization in populations of coupled
  oscillators.'' \emph{Physica D}, 143(1-4), 1-20.

  \begin{itemize}
  \tightlist
  \item
    Mathematical analysis of synchronization transitions
  \end{itemize}
\item
  \textbf{Walker, M. P. (2009).} ``The role of sleep in cognition and
  emotion.'' \emph{Annals of the New York Academy of Sciences}, 1156(1),
  168-197.

  \begin{itemize}
  \tightlist
  \item
    Neuroscience basis for sleep-inspired consolidation
  \end{itemize}
\item
  \textbf{Buzsáki, G. (2006).} \emph{Rhythms of the Brain.} Oxford
  University Press.

  \begin{itemize}
  \tightlist
  \item
    Neural oscillations and frequency band separation
  \end{itemize}
\item
  \textbf{Knuth, D. E. (1997).} \emph{The Art of Computer Programming,
  Vol. 2: Seminumerical Algorithms.} Addison-Wesley.

  \begin{itemize}
  \tightlist
  \item
    Golden ratio hash function and random number generation
  \end{itemize}
\item
  \textbf{Payopay, A. (2025).} ``Sleep-Inspired Consolidation of Fractal
  Agent Memories via Coupled Oscillator Dynamics.'' \emph{In
  preparation.}

  \begin{itemize}
  \tightlist
  \item
    Full manuscript (Paper 7) with C175 experimental validation
  \end{itemize}
\end{enumerate}

\begin{center}\rule{0.5\linewidth}{0.5pt}\end{center}

\textbf{Author:} Aldrin Payopay (aldrin.gdf@gmail.com) \textbf{License:}
GPL-3.0 \textbf{Repository:}
https://github.com/mrdirno/nested-resonance-memory-archive \textbf{Last
Updated:} 2025-10-29

\begin{center}\rule{0.5\linewidth}{0.5pt}\end{center}

\textbf{Quote:} \textgreater{} \emph{``The phase is the message.
Initialization is the seed. Transcendence is the substrate. Emergence is
the harvest.''}

\end{document}
