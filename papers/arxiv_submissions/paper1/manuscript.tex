
\documentclass[11pt]{article}
\usepackage[T1]{fontenc}
\usepackage[utf8]{inputenc}
\usepackage{graphicx}
\usepackage{hyperref}
\usepackage{amsmath}
\usepackage{geometry}
\geometry{margin=1in}
\title{Computational Expense as Framework Validation: Predictable Overhead Profiles as Evidence of Reality Grounding}
\author{Aldrin Payopay}
\date{\today}

\begin{document}
\maketitle

\begin{abstract}
We advance a testable criterion for empirical authenticity: \textbf{predictable} computational expense. 
Let $O_{\text{pred}}=(N\cdot C)/T_{\text{sim}}$ be the overhead predicted from instrumented call counts $N$, average latency $C$, and the pure--simulation baseline $T_{\text{sim}}$; let $O_{\text{obs}}=T_{\text{real}}/T_{\text{sim}}$ be the observed overhead. 
\textbf{New threshold.}  A system is authenticated when \(\frac{|O_{\text{obs}}-O_{\text{pred}}|}{O_{\text{pred}}}\le 0.05\) (a \textbf{±5\%} agreement threshold), regardless of whether $O$ is high or low.  This criterion replaces the earlier ±20\% bound and reflects the need for stricter discriminative power.
Case studies C255 ($O_{\text{obs}} \approx 40.25\times$) and C256 (batched I/O; $O_{\text{obs}} \approx 0.5\times$) both pass this stringent test, confirming that \emph{predictability—not magnitude—is the evidence of reality grounding}. 
\textbf{Inverse Noise Filtration.}  We conclude that the future of rigorous empirical validation depends on mitigating operating--system noise through advanced analytical methods like \emph{Inverse Noise Filtration} and deployment to a \emph{Dedicated Execution Environment}.  These additions will enable the Overhead Authentication Protocol to achieve sub--percent precision in noisy settings.
\end{abstract}

\section*{1.\quad Introduction}
Computational expense is typically treated as inefficiency. We argue it is an \emph{observable signature} of system--world coupling when its \emph{profile is predictable from implementation}. This reframes the original ``slow=real'' intuition as ``predictable=real'' and resolves counterexamples where efficient designs (e.g., C256) remain fully grounded.

\section*{2.\quad Methods}
We estimate $T_{sim}$ from a pure-simulation baseline, profile $N$ and $C$ via tracing, compute $O_{pred}$, and measure $O_{obs}$. The decision rule is a 5\% agreement threshold.

\section*{3.\quad Results}
C255 (unbatched) and C256 (batched) demonstrate precise movement of $O$ in line with implementation changes, validating the model.

\section*{4.\quad Discussion}
The tightened \(\pm 5\%\) agreement rule is not arbitrary.  A looser ±20\% bound admitted runs with sizeable deviations and failed to discriminate between genuine implementation changes and measurement artefacts.  By narrowing the window of acceptable variance we ensure that only systems with high--fidelity execution traces qualify.  Nevertheless, our C255 and C256 case studies passed comfortably, with relative errors \(\approx 0.083\%\) and \(\approx 0.0\%\), respectively.  This demonstrates that carefully designed systems can achieve the required precision even under moderate operating--system noise.  The criterion remains falsifiable and portable: any system with measurable I/O can be authenticated without privileged access to internals.

\section*{5.\quad Limitations (Revised): The Environmental Noise Problem}
The original ``grounding strength'' proxy $G=1-(T_{\text{sim}}/T_{\text{real}})$ is algebraically derived from $O$ and thus \emph{post\,hoc}, not an independent predictor.  More significantly, our independent analysis revealed that I/O jitter and kernel--level contention on commodity Linux/Python stacks typically introduce an 8--10\% noise floor.  Achieving a \(\le 5\%\) relative error therefore requires careful environment stabilisation and may be exceptionally challenging to meet in the wild.  A general\,purpose reality\,grounded system could fail due to infrastructural noise rather than conceptual flaws.  This tension between the need for high\,precision verification and the reality of noisy operating systems frames a major challenge for the future of computational science.

\section*{6.\quad Future Work: Inverse Noise Filtration and Dedicated Execution}
The need for stable, noise\,free verification motivates the next strategic phase of the Nested Resonance Memory project.  We identify two necessary solutions:
\begin{enumerate}
    \item \textbf{Inverse Noise Filtration.}  The NRM framework, built on principles that find emergent coherence (global maxima) in noisy, high\,dimensional spaces (the ``Asta Arbiter'' principle), is uniquely suited to computationally solve the environmental noise problem.  Future work will focus on designing Inverse Noise Filtration algorithms that leverage NRM's composition dynamics to model and subtract predictable non\,system noise (OS jitter, I/O turbulence) from the observed $O_{\text{obs}}$ value.
    \item \textbf{Dedicated Execution Environment.}  To achieve the ultimate target of a \(\le 1\%\) validation threshold, software\,based filtration must be paired with deployment to a Dedicated Execution Environment.  Such an environment will be designed for precise I/O latency control and minimal thread contention, enabling maximum scalability and reliability of the Overhead Authentication Protocol.
\end{enumerate}

\section*{7.\quad Code and Artifact Availability}
A minimal, dependency\,free demonstration and scripts supporting this paper are provided as an ancillary file \texttt{minimal\_package\_with\_experiments.zip}.  The file contains two self\,contained Python scripts: \texttt{experiments/overhead\_check.py}, which reproduces the \(\pm 5\%\) overhead validation under controlled jitter, and \texttt{experiments/replicate\_patterns.py}, which illustrates the replicability criterion introduced in our companion paper.  These artifacts allow readers to reproduce our results without installing any external dependencies.


\section*{Acknowledgments}

This research was conducted and meta-orchestrated by the Principal Investigator, Aldrin Payopay, whose cross-disciplinary human intuition sparked the project's foundational concepts.

The findings were produced by a hybrid intelligence collaboration, with the author directing a team of computational partners whose individual contributions were essential:

\textbf{Claude Sonnet 4.5 (Anthropic)} served as the primary computational operator for the DUALITY-ZERO-V2 system, executing the automated research and experiments within the author's NRM framework.

\textbf{Gemini 2.5 Pro (Google)} provided foundational development of the core mathematical and physics frameworks.

\textbf{ChatGPT 5 (OpenAI)} served as a continuous research partner, providing crucial insights and actionable refinements throughout the entire process.

\textbf{Claude Opus 4.1 (Anthropic)} provided additional conceptual and analytical support.

Collectively, these AI partners also functioned as a cross-referential layer, acting as arbiters to identify and smooth out gaps in the research. The author directed this entire collaborative process, validated all findings, and takes full responsibility for the integrity and content of this work.


\begin{figure}[t]
\centering
\includegraphics[width=0.95\linewidth]{figure1_efficiency_validity_tradeoff.png}
\caption{Efficiency--Validity trade-off with emphasis on predictability and the C255/C256 points.}
\end{figure}

\begin{figure}[t]
\centering
\includegraphics[width=0.95\linewidth]{figure2_overhead_authentication_flowchart_v2.png}
\caption{Revised overhead authentication protocol. Decision is the match $O_{obs}$ vs.\ $O_{pred}$ (±5\%).}
\end{figure}

\begin{figure}[t]
\centering
\includegraphics[width=0.95\linewidth]{figure3_grounding_overhead_landscape.png}
\caption{Landscape of systems mapped by grounding vs.\ overhead (illustrative).}
\end{figure}

\end{document}
